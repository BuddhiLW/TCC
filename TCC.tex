\documentclass[
% -- opções da classe memoir --
12pt,				% tamanho da fonte
openright,			% capítulos começam em pág ímpar (insere página vazia caso preciso)
oneside,			% para impressão em recto e verso. Oposto a oneside
a4paper,			% tamanho do papel.
% -- opções da classe abntex2 --
% chapter=TITLE,		% títulos de capítulos convertidos em letras maiúsculas
% section=TITLE,		% títulos de seções convertidos em letras maiúsculas
% subsection=TITLE,	% títulos de subseções convertidos em letras maiúsculas
% subsubsection=TITLE,% títulos de subsubseções convertidos em letras maiúsculas
% -- opções do pacote babel --
english,			% idioma adicional para hifenização
french,				% idioma adicional para hifenização
spanish,			% idioma adicional para hifenização
brazil,				% o último idioma é o principal do documento
]{abntex2}


% ---
% PACOTES
% ---

% ---
% Pacotes fundamentais
% ---
\usepackage{lmodern}			% Usa a fonte Latin Modern
\usepackage[T1]{fontenc}		% Selecao de codigos de fonte.
\usepackage[utf8]{inputenc}		% Codificacao do documento (conversão automática dos acentos)
\usepackage{indentfirst}		% Indenta o primeiro parágrafo de cada seção.
\usepackage{color}				% Controle das cores
\usepackage{graphicx}			% Inclusão de gráficos
\usepackage{microtype} 			% para melhorias de
% justificação
\usepackage{amsmath}
\usepackage{xltxtra}
% \setmainfont{Source Han Sans CN}
% \usepackage{xeCJK}
% ---
% \usepackage[UTF8]{ctex}
% \usepackage{xeCJK}
% \setCJKmainfont{SimSun}
% ---
% Pacotes de citações
%\usepackage{graphicx}
\usepackage{grffile}
\usepackage{longtable}
\usepackage{wrapfig}
\usepackage{rotating}
\usepackage[normalem]{ulem}
\usepackage{amsmath}
\usepackage{textcomp}
\usepackage{amssymb}
\usepackage{capt-of}
\usepackage{hyperref}
\usepackage{xltxtra}
\setmainfont{Source Han Sans CN}
\usepackage[brazilian,hyperpageref]{backref}	 % Paginas com as citações na bibl
\usepackage[alf]{abntex2cite}	% Citações padrão ABNT
\usepackage{graphicx}
\usepackage{graphics}

% ---
% Pacotes adicionais, usados no anexo do modelo de folha de identificação
% ---
% \usepackage{multicol}
% \usepackage{multirow}
% ---

% ---
% Pacotes adicionais, usados apenas no âmbito do Modelo Canônico do abnteX2
% ---
% \usepackage{lipsum}				% para geração de dummy text
% ---


% ---
% CONFIGURAÇÕES DE PACOTES
% ---

% Enumeração extendível
\usepackage{enumitem}
\setlist{nolistsep}

% Caminho dos arquivos-imagem
\graphicspath{
  {./Imagens/}
  {./Imagens/Linux}
  {./Imagens/WM}
  {./Imagens/BenchMarks}
  {./Imagens/Running}
}

% ---
% Configurações do pacote backref
% Usado sem a opção hyperpageref de backref
\renewcommand{\backrefpagesname}{Citado na(s) página(s):~}
% Texto padrão antes do número das páginas
\renewcommand{\backref}{}
% Define os textos da citação
\renewcommand*{\backrefalt}[4]{
  \ifcase #1 %
  Nenhuma citação no texto.%
  \or
  Citado na página #2.%
  \else
  Citado #1 vezes nas páginas #2.%
  \fi}%
% ---

% ---
% Informações de dados para CAPA e FOLHA DE ROSTO
% ---
\titulo{Softwares Livres na Academia e na Indústria}
\autor{Aluno: Pedro G. Branquinho \\ Orientador: Dr. Wei-Liang Qian, 钱卫良}
\local{Lorena, São Paulo}
\data{\today}%13 de Fevereiro de 2020}
\instituicao{%
  Universidade de São Paulo - USP \\
  Escola de Engenharia de Lorena
  \par
  Tese de Conclusão de Curso}
\tipotrabalho{Tese}
% O preambulo deve conter o tipo do trabalho, o objetivo,
% o nome da instituição e a área de concentração
% \preambulo{Relato da elaboração e progresso do minicurso de \LaTeX}
% ---

% ---
% Configurações de aparência do PDF final

% alterando o aspecto da cor azul
\definecolor{blue}{RGB}{41,5,195}

% informações do PDF
\makeatletter
\hypersetup{
  % pagebackref=true,
  pdftitle={\@title},
  pdfauthor={\@author},
  pdfsubject={\imprimirpreambulo},
  pdfcreator={Pedro G. Branquinho},
  pdfkeywords={tese}{software}{livre},
  colorlinks=true,       		% false: boxed links; true: colored links
  linkcolor=blue,          	% color of internal links
  citecolor=blue,        		% color of links to bibliography
  filecolor=magenta,      		% color of file links
  urlcolor=blue,
  bookmarksdepth=4
}
\makeatother
% ---

% ---
% Espaçamentos entre linhas e parágrafos
% ---

% % O tamanho do parágrafo é dado por:
\setlength{\parindent}{0.8cm}

% % Controle do espaçamento entre um parágrafo e outro:
\setlength{\parskip}{0.2cm}  % tente também \onelineskip

% ---
% compila o indice
% ---
\makeindex
% ---

% ----
% Início do documento
% ----
\begin{document}

% Seleciona o idioma do documento (conforme pacotes do babel)
% \selectlanguage{english}
\selectlanguage{brazil}

% Retira espaço extra obsoleto entre as frases.
\frenchspacing

% ----------------------------------------------------------
% ELEMENTOS PRÉ-TEXTUAIS
% ----------------------------------------------------------
% \pretextual

% ---
% Capa
% ---
\imprimircapa
% ---

% ---
% RESUMO
% ---

% resumo na língua vernácula (obrigatório)
\setlength{\absparsep}{18pt} % ajusta o espaçamento dos parágrafos do resumo
\begin{resumo}
  
  
  \noindent
  \textbf{Palavras-chaves}: software livre. automação. freqtrade. idústria. academia.

\end{resumo}
% ---


% ---
% inserir o sumario
% ---
\pdfbookmark[0]{\contentsname}{toc}
\tableofcontents*
% ---


% ----------------------------------------------------------
% ELEMENTOS TEXTUAIS
% ----------------------------------------------------------
\textual

% ----------------------------------------------------------
% Introdução (exemplo de capítulo sem numeração, mas presente no Sumário)
% ----------------------------------------------------------
\chapter[Introdução]{Introdução}
% \addcontentsline{toc}{chapter}{Introdução}

Na formação de um engenheiro físico, o qual, por definição, é um profissional generalista, os softwares abertos (FOSS - Free and Open Source Software) e a participação da comunidade Open Source são detrimentais para sua formação.

A diversidade os quais softwares extensíveis acaretam (\autoref{sec:diversidade}) podem mudar completamente a experiência do usuário, e o trazer mais próximo do papel de desenvolvedor. Essa experiência não necessita de ser exclusiva de ciêntistas da computação ou profissionais de TI. Pois, a programação pode ser encarada tanto como ciência e arte.

Os Softwares Abertos possuem quatro liberdades pétreas \autoref{sec:opensource}{}, garantindo os direitos de estudo, cópia, modificação e redistribuição.

Bem como a ciência se beneficia com seus rápidos avanços, de uma comunidade global de participantes, com as mais distintas especializações profissionais. Também, beneficia-se a computação com a comunidade aberta, e especialização eclética de membros e softwares.

% <<O que são os softwares livres e como se aplicam no contexto acadêmico e industrial>>

% É notório o fato de que optimizações 

\section{Objetivo}

% Demonstrar a importância dessa categoria de softwares, e da comuniadade aberta, no desenvolvimento tecnológico.

\chapter{Revisão Bibliográfica}
\section{Open Source}
\label{sec:opensource}
Qualquer programa que permita o usuário-programador ter as seguintes liberdades:

\begin{enumerate}
\item Direito de rodar o programa, como você desejar, para qualquer fim.
\item Direito ao acesso ao código-fonte, para estudá-lo.
\item Direito de cópia e distribuição.
\item Direito à modificação do software.
\end{enumerate}

De maneira prática, a comunidade Open Source, fundamentalmente, se baseia no compartilhamento de suas configurações. As vantagens de existirem inúmeras outras pessoas utilizando o mesmo software é de que a melhoria da fronteira do programa é expandida de forma acrescida, em comparação a de um time restrito de usuários.

\subsection{Diversidade}
\label{sec:diversidade}

Dado que um direito fundamental dos softwares livres é a modificação e propagação das versões modificadas, existe uma diversidade de expressividade, sem paralelos em outras áreas da tecnologia.

Por exemplo, uma parte de software fundamental na configuração de um computador é seu gerenciador de interfaces (Window Manager). Onde, um programa é devotado a gerenciar como outros programas gráficos devem se dispor na tela de computador. Enquanto sistemas operacionais (Operational Systems) privados, como Windows e MacOS possuem versões lançadas frequentemente - vinte e cinco versões lançadas de Windows \cite{} e vinte de MacOS \cite{} -, os usuários são cerciados do direito de extender ou alterar os comportamentos programados no sistema e vítimas do suporte de suas versões.

Em contra partida, existem, paralelamente, por volta de 278 distribuições de Linux \cite{}. Onde, existem as distribuições raízes, com princípios e filosofias de desenvolvimentos teóricos e práticos diferentes.

Assim, bem como em qualquer outro escopo de software, a variação dos softwares abertos e livres (FOSS) sempre serão superiores aos monopolizados.

% \begin{}

\section{O Linux}

Existem distribuições raízes de linux, das quais muitas distribuições existem como ramificações. Nomeia-se, de forma genêrica, devido aos princípios base de uma classe de distribuições, como famílias. Cita-se algumas das mais influentes e populares, Red Hat Linux, Debian, CentOS, Fedora,Pacman-based, OpenSUSE, Gentoo-based, Ubuntu-based, Slackware, Open Sourced-based e as distribuições Independetes.

É possível apreciarmos visualmente a riqueza de distribuições pela \autoref{fig:linux-genealogy}.

\begin{figure}[!htb]
  \caption{\label{fig:linux-genealogy} Genealogia Distribuições Linux}
  \includegraphics[height=\textwidth, angle=-90]{diversidade}
  \legend{Histórico de evolução das distribuições Linux \cite{}}
\end{figure}

\section{O \LaTeX}

O \LaTeX{} possui separação entra as tarefas de produção de um
documento. A linguagem permite-nos separar as tarefas de formatação do texto, da escrita de seu conteúdo. Desta forma, o usuário concentra-se
exclusivamente em seu conteúdo, em um estágio da escrita do documento. E, na formatação de sua aparência, em outro momento.

Assim, ganha-se em qualidade de produção. Bem como, ganha total autonomia sob o documento, pois a programação da disposição gráfica dos elementos textuais pode ser programada - isto é, modificada indefinidamente, a partir dos comportamentos padrões dos pacotes utilizados. O sistema tipográfico de \LaTeX{} chegou a ser considerado o sistema digital de
tipografia mais sofisticado que existe, devido a essa paradigma de
programação funcional, \textit{bottom-up} \cite{haralambous2007}.

O \LaTeX, tecnicamente, é a junção do sistema de tipografia \TeX,
inventado por Donald Knuth, para tipografia de alto nível
\cite{knuth1986}; com os poderosos macros que facilitam a extensão do programa \TeX, a qual damos o nome de
\LaTeX. O \LaTeX{} foi inicialmente desenvolvido por Leslie Lamport, com
seus pacotes fundamentais de formatação \cite{lamport1994}. O \LaTeX,
por conseguinte, não é somente uma linguagem de tipografia de alto
nível, mas também um conjunto de macros para facilitar a tipografia em
si. Qualifica-se, assim, como um sistema de preparação de documentos;
uma linguagem markup de domínio específico.

\subsection{Classe Canônica ABNT de produção científica}

Documentos sob os requisitos das normas ABNT (Associação Brasileira de Normas
Técnicas) para elaboração de documentos técnicos e científicos
brasileiros - como artigos científicos, relatórios técnicos, trabalhos
acadêmicos, como teses, dissertações, projetos de pesquisa e outros
documentos do gênero \cite{abntex2012} - é ao que se chama classe
canônica ABNT.

\begin{citacao}
  Os documentos indicados tratam-se de “Modelos Canônicos”, ou seja,
  de modelos que não são específicos a nenhuma universidade ou instituição, mas
  que implementam exclusivamente os requisitos das normas da ABNT, Associação
  Brasileira de Normas Técnicas. \cite[Cap. 1]{araujoclasse}
\end{citacao}

\clearpage

As normas as quais prescrevem o modelo canônico são:

\begin{itemize}
\item \textbf{ABNT NBR 6022:2018:} Informação e documentação -
  Artigo em publicação periódica científica - Apresentação.
\item \textbf{ABNT NBR 6023:2002:} Informação e documentação -
  Referência - Elaboração.
\item \textbf{ABNT NBR 6024:2012:} Informação e documentação -
  Numeração progressiva das secções de um documento - Apresentação.
\item \textbf{ABNT NBR 6027:2012:} Informação e documentação -
  Sumário - Apresentação.
\item \textbf{ABNT NBR 6028:2003:} Informação e documentação -
  Resumo - Apresentação.
\item \textbf{ABNT NBR 6029:2006:} Informação e documentação -
  Livros e folhetos - Apresentação.
\item \textbf{ABNT NBR 6034:2004:} Informação e documentação -
  Índice - Apresentação.
\item \textbf{ABNT NBR 10520:2002:} Informação e documentação -
  Citações.
\item \textbf{ABNT NBR 10719:2015:} Informação e documentação -
  Relatórios técnicos e/ou científico - Apresentação.
\item \textbf{ABNT NBR 14724:2011:} Informação e documentação -
  Trabalhos acadêmicos - Apresentação.
\item \textbf{ABNT NBR 15287:2011:} Informação e documentação -
  Projeto de pesquisa - Apresentação.
\end{itemize}

% \clearpage
\section{Freqtrade}

% <<No que consiste o software; os métodos utilizados para seu funcionamento>>

\section{DifferentialEquations}

Software utilizável por Julia, Python e R. É possível resolver as seguintes categorias de equações diferenciais,

% Discrete equations (function maps, discrete stochastic (Gillespie/Markov) simulations)
% Ordinary differential equations (ODEs)
% Split and Partitioned ODEs (Symplectic integrators, IMEX Methods)
% Stochastic ordinary differential equations (SODEs or SDEs)
% Stochastic differential-algebraic equations (SDAEs)
% Random differential equations (RODEs or RDEs)
% Differential algebraic equations (DAEs)
% Delay differential equations (DDEs)
% Neutral, retarded, and algebraic delay differential equations (NDDEs, RDDEs, and DDAEs)
% Stochastic delay differential equations (SDDEs)
% Experimental support for stochastic neutral, retarded, and algebraic delay differential equations (SNDDEs, SRDDEs, and SDDAEs)
% Mixed discrete and continuous equations (Hybrid Equations, Jump Diffusions)
% (Stochastic) partial differential equations ((S)PDEs) (with both finite difference and finite element methods)

\section{OR-Tools}
\label{sec:ortools}

Explicar a utilidade do ortools; utilizável em C++, Python, Java, (Clojure), C\#, .Net.

% \section{Beamer}

% Desenvolvido pela comunidade, Beamer não é a primeira, porém, o mais
% utilizado pacote para produção de slides e apresentações. Seus
% desenvolvedores, iniciais, foram Louis Stuart, Till Tantau, Joseph
% Wright, e Vedran Miletc \cite{tantau2010}. Por mais que estes sejam os principais
% desenvolvedores, Beamer é um pacote livre e aberto, como o \LaTeX, em
% si. Isto é, toda a comunidade usuária é também desenvolvedora do
% pacote.


% Desta forma, é um pacote extensivamente trabalhado para
% produção de apresentações, disponível em diversas formatações
% canônicas. A \autoref{im:10} apresenta dois exemplos de slides feitos
% com Beamer, encontrados na internet.

% \begin{figure}[!htb]
%   \caption{\label{im:10} Exemplos de Slides Produzidos com Beamer}
%   \begin{center}
%     \includegraphics[width=0.45\linewidth]{./Imagens/beamer.png}
%     \includegraphics[width=0.45\linewidth]{./Imagens/beamer2.jpg}
%   \end{center}
%   \legend{Fonte: \href{https://martin-thoma.com/author/martin-thoma/}{\cite{martin2013}}}
% \end{figure}



% \section{Tikz}

% Tikz é um pacote essencial para manusear e crear imagens com
% Postscript, por meio da linguagem \TeX. Com essa ferramenta, pode-se
% fazer diagramas, gráficos, desenhos. Ele é um dos pacotes fundamentais
% do \LaTeX{} \cite{lamport1994}. As imagens - \autoref{im:1}, \autoref{im:2},
% \autoref{im:3}, \autoref{im:4}, \autoref{im:5}, \autoref{im:6}  - apresentam explanações de
% código/resultado encontrados na documentação oficial do pacote \href{http://linorg.usp.br/CTAN/graphics/pgf/base/doc/pgfmanual.pdf}{Tikz}.

% \begin{figure}[!htb]
%   \caption{\label{im:1} Imagens escritas com Tikz, geometria}
%   \begin{center}
%     \includegraphics[width=\linewidth]{./Imagens/3.png}
%   \end{center}
%   \legend{Fonte:
%     \href{http://linorg.usp.br/CTAN/graphics/pgf/base/doc/pgfmanual.pdf}{Manual
%       Tikz, CTAN}}
% \end{figure}

% \begin{figure}[!htb]
%   \caption{\label{im:2} Imagens escritas com Tikz, grafos}
%   \begin{center}
%     \includegraphics[width=\linewidth]{./Imagens/1.png}
%   \end{center}
%   \legend{Fonte:
%     \href{http://linorg.usp.br/CTAN/graphics/pgf/base/doc/pgfmanual.pdf}{Manual
%       Tikz, CTAN}}
% \end{figure}
% %% \begin{figure}[!htb]
% %%   \caption{\label{im:3} Imagens escritas com Tikz, gráficos}
% %%   \begin{center}
% %%     \includegraphics[width=0.47\linewidth, \height=0.4\paperheight]{./Imagens/2.png}
% %%     \includegraphics[width=0.47\linewidth, \height=0.4\paperheight]{./Imagens/6.png}
% %%   \end{center}
% %%   \legend{Fonte:
% %%   \href{http://linorg.usp.br/CTAN/graphics/pgf/base/doc/pgfmanual.pdf}{Manual
% %%   Tikz, CTAN}}
% %% \end{figure}

% \begin{figure}[!htb]
%   \caption{\label{im:3} Imagens escritas com Tikz, gráficos}
%   \begin{center}

%     \includegraphics[width=\linewidth, \height=0.4\paperheight]{./Imagens/6.png}
%   \end{center}
%   \legend{Fonte:
%     \href{http://linorg.usp.br/CTAN/graphics/pgf/base/doc/pgfmanual.pdf}{Manual
%       Tikz, CTAN}}
% \end{figure}

% \begin{figure}[!htb]
%   \caption{\label{im:4} Imagens escritas com Tikz, gráficos}
%   \begin{center}
%     \includegraphics[width=\linewidth, \height=0.4\paperheight]{./Imagens/2.png}
%   \end{center}
%   \legend{Fonte:
%     \href{http://linorg.usp.br/CTAN/graphics/pgf/base/doc/pgfmanual.pdf}{Manual
%       Tikz, CTAN}}
% \end{figure}



% \begin{figure}[!htb]
%   \caption{\label{im:5} Imagens escritas com Tikz, representação da Luz}
%   \begin{center}
%     \includegraphics[width=\linewidth]{./Imagens/4.png}
%   \end{center}
%   \legend{Fonte:
%     \href{http://linorg.usp.br/CTAN/graphics/pgf/base/doc/pgfmanual.pdf}{Manual
%       Tikz, CTAN}}
% \end{figure}


% \begin{figure}[t]
%   \caption{\label{im:6} Imagens escritas com Tikz, representação de um
%     arbusto}
%   \begin{center}
%     \includegraphics[width=\linewidth]{./Imagens/5.png}
%   \end{center}
%   \legend{Fonte:
%     \href{http://linorg.usp.br/CTAN/graphics/pgf/base/doc/pgfmanual.pdf}{Manual
%       Tikz, CTAN}}
% \end{figure}

\clearpage



\chapter{Materiais e Métodos}

% O planejamento do minicurso seguiu as seguintes etapas:

% \begin{enumerate}
% \item Determinação do local/ambiente a serem ministradas as aulas presenciais e a distância;
% \item Capacidade de vagas para as aulas;
% \item Datas para realização das aulas;
% \item Definição de temas e carga horária para cada aula;
% \item Recursos e datas para divulgação e inscrição.
% \item Elaboração de materiais e atividades para execução do curso;
% \item Recursos para administração do minicurso;
% \item Execução das aulas;
% \item Validação da participação dos alunos;
% \item Emissão de certificados.
% \end{enumerate}

% Porém, quando houve o anúncio da quarentena, pela USP, devido ao
% Covid-19, planejou-se um novo cronograma e escolheu-se as plataformas,
% para aulas à distância.

% Decidiu-se por aulas disponibilizadas no Youtube, e encontros pela
% plataforma Google Meet.

% \section{Sediação}
% O curso foi idealizado para até quarenta alunos. Procurou-se uma sala
% que os comportasse, bem como tivesse os devidos equipamentos
% eletrônicos para sediar apresentações.

% Por fim, no entanto, as aulas foram gravadas e disponibilizadas no
% Youtube. Foi, ademais, dadas aulas de revisão e dúvidas, pela
% plataforma Google Meet. Da mesma forma, a disponibilidade também foi
% readequada às condições; abriu-se vagas para um número de até cem
% alunos. Pois, não havia a limitação de espaço físico.

% \section{Data}
% Procurou-se fazer escolha de datas que maximizassem a aderência, e
% notoriedade do curso. Levou-se em conta, também, o calendário oficial
% da USP, do semestre letivo.

% \section{Definição dos temas}

% As divisões de aula do curso foram feitas, esquematicamente, em nível
% progressivo de complexidade e abstração, requerida para se produzir
% documentos com alocação de imagens e modulações, como apresentações e pôsteres.

% Procurou-se, para a escolha dos tópicos, e partição das aulas,
% entender qual eram as necessidades dos alunos, quanto a produção
% acadêmicos. Ao mesmo tempo, o curso foi projetado de forma a maximizar o
% aprendizado sistemático do assunto.  Determinou-se que, tanto a
% formatação sob as normas ABNT, como a formatação de apresentações de slides eram essenciais.

% \section{Divulgação}

% Procurou-se determinar quais eram os canais mais populares, e com
% maiores retornos de público, para se divulgar. Bem como alocar a
% divulgação em datas estratégias.

% % E sobre os criterios de carga horária, avaliação, execução e conclusão com emissão dos certificados?
% % Também não escreveu sobre as inscrições....

% \section{Carga horária}
% Para cada aula, foi alocada duas horas aula. Duas a quatro horas para se estudar
% os materiais de cada módulo. E, duas a quatro horas para se completar cada lista
% de exercício. Assim, o curso requereria, no mínimo, 8 horas-crédito de
% dedicação para ser realizado.

% \subsection{Plantões de Dúvida}
% Ademais, o aluno teve abertura num horário específico para se
% encontrar com o professor, revisar a matéria, e sanar dúvidas. Essa
% carga horária não seria detrimental à conclusão do curso, configurando
% um recurso extra de aprendizado. Houve um plantão por
% aula. Configurando-se três plantões de duração de 1h, cada.

% \section{Inscrição}

% As inscrições seriam feitas por meio da internet, com a utilização do Google Forms.
% Com esse recurso, coletar-se-iam número USP do aluno(a), telefone, e e-mail.

% \section{Avaliação}
% Cada aula contaria com uma lista de exercícios, como método avaliativo. Desta forma,
% haveriam três listas as quais contariam pela aprovação ou reprovação dos alunos.

% \section{Conclusão do Curso}

% Ao fim do curso, seria dado ao aluno o qual completasse as listas um certificado
% de participação e conclusão. Não se avisou os alunos sobre a gratificação, de
% acordo com os conhecimentos e resultados decorrentes dos estudos de motivação
% dos seres humanos \cite{deci2017self}. Fazer de outra maneira,
% seria benéfico à participação das atividades \cite{hendijani2016intrinsic}.
% Porém, detrimental ao desempenho posterior do aluno
% \cite{ariely2009large}.

% \section{Formulário de Satisfação}
% Após conclusão do curso, propõe-se avaliar o nível de satisfação do
% aluno. Nisso, consiste um formulário para dar abertura ao aluno
% retornar sua avaliação do professor, do curso e suas recomendações de
% mudanças.

\chapter{Resultado e Discussões}

% \section{Sediação}
% Inicialmente a proposta foi para aulas presencias e definiu-se a sala EF-15, nas instalações do DEMAR, a qual apresentava capacidade de suportar 40 alunos. Porém, devido à
% suspensão das atividades presenciais pela USP, por causa da pandemia de
% COVID-19, as aulas presenciais planejadas na sala EF-15 foram
% substituídas por aulas a distância com uso de recursos
% remotos.

% \section{Data}
% De acordo com as datas do calendário USP, obtido na plataforma
% Jupiterweb, as aulas começam uma semana antes do carnaval. E, são
% demarcadas como semana de apresentação e recepção dos calouros - 17 à
% 21 de Janeiro de 2020. Assim,
% decidiu-se, estrategicamente, alocar o início do minicurso para depois
% do carnaval, dia 27 de Fevereiro de 2020. No fim, as aulas à distância começaram em Junho, dia
% 1.

% O cronograma do trabalho, \autoref{im:7}, apresenta como todo o curso
% foi planejado e dado, mesmo com a quarentena, incidente por causa do
% Covid-19. Assim, o curso foi ministrado integralmente remoto.

% \begin{figure}[!htb]
%   \caption{\label{im:7} Cronograma PUB, minicursos}
%   \begin{center}
%     \includegraphics[scale=0.27]{./Imagens/9.png}
%   \end{center}
%   \legend{Fonte: O autor}
% \end{figure}



% \clearpage

% \section{Definição dos temas do minicurso}

% Todo o minicurso foi ministrado num modelo de
% Ensino à Distância.

% \subsection{Produção do Material Integral}

% Durante o período de Agosto a Janeiro, desenvolveu-se as apresentações, e
% uma apostila virtual, a qual contém material suplementar de estudo e
% referência. A apostila foi disponibilizada por meio do
% \href{https://github.com/26-55-87-BuddhiLW/MC-LaTeX}{repositório no
%   GitHub}, e Google Clasroom, bem como materiais de apoio,
% exemplos, e os modelos canônicos ABNT usados nas aulas.  Usou-se as
% plataforma do Youtube, no canal da
% \href{https://www.youtube.com/channel/UC9kL6UL-iEq4nJKGs1nezQQ}{LabEEL}
% para disponibilizar as aulas à distância.

% Com o software Krita, criou-se os planos de fundo que foram utilizados
% pelo autor nas apresentações. Para isso, utilizou-se logos, e
% logomarcas, da USP, da EEL, e do grupo LabEEL.


% \subsection{Temas}

% \setlist[enumerate]{noitemsep, lineskip=0pt}
% \begin{enumerate}
% \item Introdução, Filosofia, e Instalação do LaTeX - utilização de templates TeX.
% \item Produção de Relatórios, Teses e Monografias, Sob Norma ABNT -
%   pacote ABNTeX, imagens, tabelas, referências bibliográficas, e
%   fórmulas matemáticas.
% \item  Apresentações com pacote Beamer, citações ABNT - controle e modulação dos parâmetros de pacotes.
% \end{enumerate}


% A sequência é lógica, pois, não seria possível explicar a utilização de qualquer
% pacote, sem o aluno saber o mínimo sobre a sintaxe da língua. Bem como, seria
% misteriosa a escolha da língua, em relação a qualquer outro software tipográfico. Assim,
% o primeiro item abordado foi a filosofia da linguagem e utilização de templates.
% Desta forma, o aluno inicia a utilização da língua; entende seus porquês de ser,
% e aprende a sintaxe básica a todo documento, escrito em \LaTeX.

% Em seguida, o aluno já tem capacidade de se aprofundar na tipografia de documentos, sem muita
% estilização específica. Por conseguinte, ensina-o a utilizar pacotes
% essenciais a um estudante universitário - i.e., abntex2.

% Por fim, as apresentações e autoformatação de citações foram
% deixadas para a última aula. Porque, para serem compreendidas,
% apresentam um nível superior de abstração em relação aos assuntos
% supramencionados.

% Ao concretizar-se o curso, o aluno adquiriu conhecimentos suficientes para entender, virtualmente, qualquer
% pacote de \LaTeX{} mais complexo, ou modelos os quais utilizem muita
% modulação, como seria o caso da produção de documentos como pôsteres.




% \section{Divulgação}

% No mês de fevereiro a partir do dia 17, utilizou-se de mídias sociais e de correspondência, como Facebook,
% Whatsapp, e e-mail USP, para que os alunos sejam informados do curso,
% bem como anunciar o período de inscrição. Escolheu-se o período de
% divulgação com base nas datas de feriados e início de aulas.

% Com o software Krita, criou-se uma arte conceptiva, para divulgação do
% curso. Utilizou-se, para essa produção, do logo oficial do LabEEL.

% A divulgação das aulas à distância, e mudança das  novas datas, foi feita no
% período de Maio.

% \section{Carga horária}
% %% Para cada aula, foi alocada duas horas aula. Duas a quatro horas para se estudar
% %% os materiais de cada módulo. E, duas a quatro horas para se completar cada lista
% %% de exercício. Assim, o curso requeriria, no mínimo, 8 horas-crédito de dedicação para ser realizado.
% Como o resultado foi de um abandono do curso foi menor do que 5\%, acredita-se que a carga horário foi coerente.

% \subsection{Plantões de Dúvidas}
% Nos horários de encontro, extra, com o professor, notou-se pouca
% aderência. Em média de três à quatro alunos compareceram por plantão
% de dúvida. Porém, as dúvidas, em geral, eram todas de aprofundamento,
% do que revisão. A pouca aderência, acredita o professor, deve-se à
% procura dos estudantes por outra mídias e horários - diversos alunos
% procuraram o professor em particular para sanar suas dúvidas.

% \section{Inscrição}
% %% As inscrições seriam feitas por meio da internet, com a utilização do Google Forms.
% %% Com esse recurso, coletaria-se número USP do aluno(a), telefone, e e-mail.

% Houveram 80 inscritos. Assim, considera-se que o curso houve muitos inscritos, provando válida a maneira com que se fez as divulgações e os meios pelos quais se disponibilizou as inscrições.

% \section{Avaliação}
% %% Cada aula contaria com uma lista de exercícios, como método avaliativo. Desta forma,
% %% haveriam três listas as quais contariam pela aprovação ou reprovação dos alunos.
% Todo aluno que fez a primeira atividade terminou por fazer as demais. As avaliações tiveram notas foram altas; mais do que 80\% da nota total, em média. Isto pois os alunos apresentaram exatamente o que foi pedido nas litas de exercícios, com algumas exceções. As exceções se dão a não entrega total da lista, ou erros de interpretação do que foi pedido.

% \section{Conclusão do Curso}

% Na conclusão do curso, percebeu-se um entusiasmo grande dos alunos, em receberem seus certificados. E, houveram procuras posteriores dos alunos para se fazer da matéria. Esses dados estão de acordo com a teoria mais aceita da literatura, sobre motivação humana \cite{hendijani2016intrinsic}.

% %% Ao fim do curso, seria dado ao aluno o qual completasse as listas um certificado
% %% de participação e conclusão. Não se avisou os alunos sobre a gratificação, de
% %% acordo com os conhecimentos e resultados decorrentes dos estudos de motivação
% %% dos seres humanos \cite{deci2017self}. Fazer de outra maneira,
% %% seria benéfico à participação das atividades \cite{hendijani2016intrinsic}.
% %% Porém, detrimental ao desempenho posterior do aluno \cite{ariely2009large}.

% \section{Participação dos Alunos}

% Houveram 80 inscritos. Desses, 78 aceitaram o
% convite de participação do Google Classroom. Apenas 22 alunos
% completaram a primeira lista de exercícios, por mais que, todos que
% fizeram, fizeram com excelência. Desses alunos, 20 fizeram a segunda
% lista. E, por fim, a terceira lista foi feita por 19 alunos.

% Assim, apenas 27.50\% dos alunos inscritos acometeram-se às atividades
% do curso, de início; 23.75\%  completaram o curso. Assim, houve uma
% desistência de 13.63\% dos participantes ativos
% Isto é, \\
% $\big(1-\frac{\text{(Alunos Completaram)\%}}{\text{(Alunos Inscritos
%     Ativos)\%}} \big) \times 100\% = (1 - \frac{23.75\%}{27.50\%}) \times 100\% = 13.63\%$.

% O resultado é compreensível, pois o curso exige muito do aluno, que
% não está acostumado com programação. Outra coisa é que o curso não é
% fácil, quem não está disposto, por
% inclinações próprias, ou por não ter tempo suficiente, não iria
% conseguir assistir às aulas, estudar o material, e completar as listas
% de exercícios - dos quais eram simulações de situações recorrentes na
% utilização do \LaTeX. Por conseguinte, observou um sucesso com os
% objetivos do minicurso de capacitação, porém, houve uma distinção brusca entre as
% partições dos alunos que se dedicaram integral às atividades, e os que
% não.



% \section{Formulário de Satisfação}

% No formulário de satisfação, as perguntas estavam distribuidas em uma
% escala de um ao cinco. A última questão era textual, em que poderia se
% dissertar sobre a qualidade do curso.

% Os alunos respoderam às perguntas,

% \begin{enumerate}
% \item ``Com
% qual nível de segurança você se sente, para utilizar o LaTeX
% profissionalmente?'' Média: 3.57, Desvio Padrão: 0.98.

% \item ``Qual foi o nível de preparo do Professor, para ministrar as aulas?''
% Média:4.71, Desvio Padrão: 0.49.

% \item ``Qual foi o nível do preparo do Professor, quanto ao material didático?''
% Média:4.71, Desvio Padrão: 0.49.

% \item ``Qual foi seu nível de engajamento no curso?''
% Média:4.0, Desvio Padrão: 0.58.

% \item ``O que você gostaria que tivesse sido diferente?''
% Apenas dois alunos responderam,
% ``Não mudaria nada. Os materiais fornecidos eram ricos em exemplos que
% me ajudaram muito e as aulas expositivas sempre ficaram muito
% claras.''; ``Não, o professor foi muito paciente e atencioso.''
% \end{enumerate}
\chapter{Conclusão}

% O objetivo de capacitação de alunos da EEL-USP foi atingido. Porém,
% com um decaimento expressivo de número de participantes ativos e
% inscritos. 27.50\% dos alunos inscritos acometeu-se às atividades
% do curso; 23.75\%  completaram o curso; houve uma
% desistência de 13.63\% dos participantes ativos.

% As listas possuíam uma complexidade coerente com os materiais e aulas
% fornecidas. De forma que, se o aluno completasse as listas, seria
% possível, seguramente, que o aluno conseguisse integrar o uso da
% ferramenta na produção de documentos, no seu dia a dia. Pois, todos os
% exercícios foram tirados de situações reais, do qual o autor
% enfrentou, em seu caminho acadêmico na faculdade.

% Ademais, recebeu-se láureos dos concluintes, por meio das mídias
% sociais, bem como com o resultado do formulário de satisfação. Os alunos
% não souberam dizer algo que gostariam de se alterar no curso. Assim,
% concluímos que, na perspectiva, também, do aluno, o curso atingiu seu objetivo.



\bibliography{bib}


\end{document}

%%% Local Variables:
%%% mode: latex
%%% TeX-master: t
%%% End:
