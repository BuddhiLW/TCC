%%% Local Variables:
%%% mode: latex
%%% TeX-master: "Branquinho"
%%% TeX-engine: xetex
%%% End:

\documentclass[landscape]{standalone}
\usepackage[utf8]{inputenc}
\usepackage{pgfgantt}
\usepackage{xcolor}
\usepackage{graphicx}
\usepackage[portuguese]{babel}
\usepackage{tikz}
\usepackage{lipsum}

\renewcommand{\rmdefault}{ptm}
% \usepackage{fontspec}
% \setmainfont{QTChanceryType}
% \usepackage[scaled=4pt]{helvet}
\usepackage{lmodern}
\usepackage[familydefault]{Chivo} %% Option 'familydefault' only
% \usepackage{emerald}
%% if the base font of the
%% document is to be sans
%% serif
% \usepackage{courier}
% \normalfont % in case the EC fonts aren't available
\usepackage[T1]{fontenc}

% \usepackage[margin=4cm, scale=1, centering,]{geometry} % to
% change margins
\usepackage[total={10 in,10 in},top=1.6in, left=0.9in]{geometry}
\usepackage{pdflscape} % provides the landscape environment
\usepackage{lscape}
\usepackage{ragged2e} % provides \RaggedLeft

\parskip=2pt\parindent 0pt


\usepackage{fontspec}

% new font from google fonts
\setmainfont[Path=fonts/Barlow/,
Extension=.ttf,
UprightFont=*-Regular
]{Barlow}

% \definecolor{orchid}{RGB}{218, 112, 214}
% \definecolor{purple1}{RGB}{144, 5, 133}
% \definecolor{orange1}{RGB}{255, 165, 0}

% \definecolor{brisa1}{RGB}{225,0,0}
% \definecolor{brisa2}{RGB}{225, 255, 0}

% \definecolor{mesa1}{RGB}{188,218,221}
% \definecolor{mesa2}{RGB}{250, 250, 250}

% \definecolor{louça1}{RGB}{200, 92, 145}
% \definecolor{louça2}{RGB}{200,92,192}

% \definecolor{sid1}{RGB}{255, 127, 0}
% \definecolor{sid2}{RGB}{0, 128, 255}

\definecolor{branqs1}{RGB}{222, 111, 222}
\definecolor{branqs2}{RGB}{222, 111, 222}



\newenvironment{myfont}{\fontfamily{pnc}\selectfont}{\par}
\newenvironment{myfont2}{\fontfamily{qhv}\selectfont}{\par}


\begin{document}

% \begin{landscape}

\begin{center}
  \begin{figure}[!bth]
    \begin{center}
      \begin{ganttchart}[
        vgrid={*{3}{black!55, thick, dashed}, *{1}{purple, thick}},
        hgrid={1*{grey}},
        title/.append style=%
        {fill=blue!12, rounded corners=0.4mm, drop shadow},
        title label font=\color{black!90}\bfseries,
        title height=0.8,
        title top shift=0.2,
        % title left shift=0.2,
        % title right shift=-0.1,
        chart element start border=right]
        {1}{35}
        \gantttitle[title/.style={draw=black,
          fill=white}]{\LARGE{\textbf{\selectfont\ubuntu
              Abril}}}{4}
        \gantttitle[title/.style={draw=black,
          fill=white}]{\LARGE{\textbf{\selectfont\ubuntu
              Maio}}}{31}\\
        % \gantttitle[title/.style={draw=black,
        % fill=white}]{\LARGE{\textbf{\fontfamily{put}\selectfont
        % Junho}}{37}{67}
        % \gantttitle[title/.style={draw=black,
        % fill=white}]{\LARGE{\textbf{\fontfamily{put}\selectfont
        % Julho}}}{68}{99}
        

        % % Numbering title
        % \begin{myfont2}
        %   \gantttitle{Sem. 1}{7} \gantttitle{Sem. 2}{7}{14}
        %   \gantttitle{Sem. 3}{7}{14} \gantttitle{Sem. 4}{7}{14}
        %   \gantttitle{Sem.5}{2}{4} \\
        % \end{myfont2}
        \begin{myfont}
          \gantttitlelist{27,...,30}{1}
          \gantttitlelist{1,...,31}{1} \\[grid]
        \end{myfont}
        %%%%%%%%%%%%%%% 
        %%%%%%%%%%%%%%% 

        % \ganttgroup[group/.style={draw=black, outer
        % color=branqs1, inner
        % color=branqs2!50}]{{\textsc{{Completo}}}}{0}{34} \\
        % +++
        \ganttgroup[group/.style={
          draw=black!80, outer
          color=purple!35,
          inner color=purple!55}]
        {{\Large{\textsc{{Completo}}}}}{0}{35} \\

        % \ganttbar[bar/.style={draw=black, outer
        % color=red!70, inner color=red!60}]{Quarentena}{20}{34}\\

        \ganttbar[bar/.style={draw=black, outer
          color=red!70, inner color=red!60}]{{\large{\textbf{Contato com Orientador}}}}{0}{4}
        \ganttnewline[thick, black]


        %%%%%%%%%%%%%%%%% %%%%%%%%%%%%%%%%%%%%%%%%%% %%%%%%%%%%%%%%%% 

        %%%%%%%%%%%%%%%%% %%%%%%%%%%%%%%%%%%%%%%%%%% %%%%%%%%%%%%%%%% 

        %%% PRIMEIRO CICLO %%%%%%%
        \ganttgroup{\large{Esquemática TCC 1.0}}{4}{13} \\
        \ganttbar[bar/.style={
          draw=black, outer
          color=orange!50,
          inner color=orange!50}]{\large\textbf{Literatura}}{4}{8}\\

        \ganttbar[bar/.style={draw=black, outer
          color=green!50, inner color=green!35}]{Organização e escrita}{8}{10}\\

        \ganttbar[bar/.style={draw=black, outer
          color=cyan!30, inner color=cyan!30}]{Anotações Org}{5}{13} \\

        % % %%% Relatório da primeira aula %%%%%%%%%
        % \ganttbar[bar/.style={draw=black!80, outer
        %   color=black!80, inner color=black!70}]{{Relat. Aula
        %     1}}{4}{13} \ganttnewline[thick, black]


        %%%%%%%%%%%%% SEGUNDO CICLO %%%%%%%%%%%%%%%
        \ganttgroup{\large{Esquemática TCC 2.0}}{8}{17} \\
        \ganttbar[bar/.style={draw=black, outer
          color=orange!50, inner color=orange!50}]{\large\textbf{Literatura}}{8}{12}\\

        \ganttbar[bar/.style={draw=black, outer
          color=green!50, inner color=green!35}]{Organização e escrita}{12}{14}\\

        \ganttbar[bar/.style={draw=black, outer
          color=cyan!30, inner color=cyan!30}]{{Anotações Org}}{9}{17} \\
        %%%%%%%%%%%%%%%%%%%%%%%%%%%%%%%%%%%%%%%%%%%%%%%% 


        % %%% Relatório da Aula 2 %%%%%%%%%
        % \ganttbar[bar/.style={draw=black!80, outer
        %   color=black!75, inner color=black!68}]{{Relat. Aula 2}}{8}{17}\ganttnewline[thick, black]
        %%%%%%%%%%%%%%%%%%%%%%%%%%%%%%%%%%%%%%%%%%%%%%%%%%%%%% 

        %%%%%%%%%%%%% TERCEIRA AULA %%%%%%%%%%%%%%%
        \ganttgroup{\large{Esquemática TCC 3.0}}{12}{21} \\
        \ganttbar[bar/.style={draw=black, outer
          color=orange!50, inner color=orange!50}]{{\large{\textbf{Aula 3}}
          }}{12}{16}\\

        \ganttbar[bar/.style={draw=black, outer
          color=green!50, inner color=green!35}]{{Organização e escrita}}{16}{18}\\

        \ganttbar[bar/.style={draw=black, outer
          color=cyan!30, inner color=cyan!30}]{{Anotações Org}}{13}{21} \\
        %%%%%%%%%%%%%%%%%%%%%%%%%%%%%%%%%%%%%%%%%%%%%%%% 

        % %%% Relatório da Aula 3 %%%%%%%%%
        % \ganttbar[bar/.style={draw=black!80, outer
        %   color=black!75, inner color=black!68}]{{Relat. Aula
        %     3}}{12}{21} \ganttnewline[thick, black]
        % %%%%%%%%%%%%%%%%%%%%%%%%%%%%%%%%%%%%%%%%%%%%%%%%%%%%%% 

        %%%%%%%%%%%%%% Relatório Final %%%%%%%%%%%%
        \ganttbar[bar/.style={draw=black!80, outer
          color=purple!75, inner color=purple!68}]{{\textbf{\Large{Comunicação de Estratégia}}}}{21}{35}

        %%%%%%%%%%%%%%%%%%%% 

        %%%%%%%%%%%%%%% 
        %%%%%%%%%%%%%%% 
      \end{ganttchart}
    \end{center}
  \end{figure}
\end{center}

% \end{landscape}
\clearpage
% \begin{landscape}

% \end{landscape}


\end{document}
%%% Local Variables:
%%% mode: latex
%%% TeX-master: "Branquinho"
%%% TeX-engine: xetex
%%% End: